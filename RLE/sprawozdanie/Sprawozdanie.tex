\documentclass{article}
\usepackage{a4wide}


\usepackage{polski}
\usepackage[utf8x]{inputenc}
\usepackage{graphicx}
\usepackage{float}
\usepackage{hyperref}
\usepackage{listings}
\usepackage{mathtools}
\usepackage{amsmath}
\usepackage{hyperref}
\usepackage[margin=1in]{geometry}
\usepackage[dvipsnames]{xcolor}


\author{Lev Sergeyev}
\title{ZAPDC \\ Ćwiczenie 7 \\ Algorytm RLE}

\date{ }
\begin{document}

\maketitle

%\pagebreakmovie15

\section{Przebieg ćwiczenia}
Zaprojektowałem program pozwalający na kompresowanie/dekompresowanie plików binarnych za pomocą algorytmu RLE. \\
Program pozwala na wybór rozmiaru podstawowego bloku danych, do wyboru 1,2,4 lub 8 bitów (co odpowiada zmiennym typu uint8, uint16, uint32, uint64). \\
Aby zakodować plik należy uruchomić program z wierszu poleceń z parametrem 'c', 'c1', 'c2', 'c4' lub 'c8' (liczba wskazuje na długość bloku w bajtach) następnie podać nazwę pliku wejściowego i wyjściowego. \\ Proces kompresowania można rozbić na takie etapy:
\begin{itemize} 
	\item Wczytanie pliku do kompresji.
	\item Analiza wczytanych danych:
	\begin{itemize} 
		\item tworzenie słownika bloków,
		\item bloków w słowniku w taki sposów, aby najczęściej wystepowany blok przy kodowaniu był na pierwszym miejscu.
	\end{itemize}
	\item Kodowanie danych - przekształcenie na ciąg kodów bloków z liczbą występowania
	\item Zapis zakodowanych danych do pliku binarnego
\end{itemize}
\par
Skompresowany plik ma następującą strukturę:
\begin{itemize} 
	\item Nagłówek wskazujący długość bloku.
	\item Tablica kodów.
	\item Ciąg zakodowanej informacji.
\end{itemize}

\par
Proces kodowania, w zależności od priorytetu bloku i długości ciagu, może zapisać do pliku wyjściowego dane o róźnym rozmiarze, np blok, który ma największy priorytet występuje w ciągu 4 takich bloków, w pliku wyjściowym będzie mieć zapis \( 0b \textcolor{violet}{10}  \textcolor{OliveGreen}{00 01 00}\) (na fioletowo - bity kodujące numer bloku, zielone bity kodują długość ciagu). \\
Innym przykładem może służyć blok, o numerze 5, i długością ciągu 205, ponieważ liczba długości nie może być zapisana w 6icu bitach, będzie ona kodowana w inny (dłuższy) sposób :  \( 0b 0001 \textcolor{violet}{0101} ,    0b \textcolor{OliveGreen}{11001101} \) . \\
Algorytm pozwala na zapis długości ciagu zarówno jak i kodu bloku o maksymalnej długości 32 bity. \\
\par
Aby zdekodować plik należy uruchomić program z poleceniem 'd'. Dekodowanie odbywa się w następujący sposów:
\begin{itemize} 
	\item Odczyt nagłówku, który zawiera informacje o rozmiarze bloku.
	\item Odczyt tablicy kodów.
	\item Odtworzenie danych do tablicy kodów z długością.
	\item Odtworzenie pliku oryginalnego.
\end{itemize}


\pagebreak
\section{Kompresja}
Aby sprawdzić działanie algorytmu na obrazach, skonwertowałem pliki wdo formatu .bmp przedstawiające bitmapy obrazów. Następnie przeprowadziłem kompresowanie dla każdego trybu. \\
Dodatkowo dołączyłem do porównania plik tekstowy.

\begin{center}
    \begin{tabular}{ | c |  p{3.4cm}|  p{2cm}| p{2cm} | p{2cm} | p{2cm} |}
    \hline
    Plik & Oryginał &  Komprescja c1 &   Komprescja c2 &   Komprescja c4 &   Komprescja c8 \\ \hline
    
    \raisebox{-\totalheight}{\scalebox{0.3}{\includegraphics{../Lysy-losowy.png}}} & Lysy-losowy.bmp  284.2 KiB & 424.0 KiB & 342.8 KiB & 268.1 KiB & 234.9 KiB
    
     \\ \hline    
    \raisebox{-\totalheight}{\scalebox{0.3}{\includegraphics{../Lysy-wzorcowy.png}}} & Lysy-wzorcowy.bmp  284.2 KiB & 58.8 KiB & 60.9 KiB & 65.6 KiB & 90.0 KiB 
    
     \\ \hline    
    \raisebox{-\totalheight}{\scalebox{0.06}{\includegraphics{../Troll-face-4-RLE.png}}} & Troll-face-4-RLE.bmp  763.0 KiB & 70.2 KiB & 83.1 KiB & 84.6 KiB & 91.3 KiB
    
     \\ \hline    
    Tekst & Lorem\_ ipsum.txt  143.4 KiB & 215.9 KiB & 190.8 KiB & 137.9 KiB & 127.4 KiB
    
    \\ \hline
    \end{tabular}
\end{center}



\subsection{Kod}
\href{https://github.com/221349/ZAPDC/tree/master/RLE}{https://github.com/221349/ZAPDC/tree/master/RLE}

\section{Wnioski}
\par
Algorytm Run-Length Encoding jest jednym z najprostrzych algorytmów kompresji bezstratnej. \\
\par
Najlepiej taki algorytm sprawdza się na prostych rysunkach mających duże obszczary tego samego koloru (jak plik "Troll-face-4-RLE.bmp") lub powtarzających się wzorów (jak plik "Lysy-wzorcowy.bmp"). \\
Kompresja plików o zawartości losowej na odwrót może spowodować zwiększenie pliku, jak w przypadku "Lysy-losowy.bmp". \\
Kompresja tekstu może dać dobry wynik, ale dla większych bloków danych, od 4 znaków.
\par
Wykryte błędy programu: kiedy illość unikalnych bloków danych jest maksymalna (np 256 dla 'c1'), najczęściej w przypadku plików z danymi losowymi, dekompresja kończy się błędem "seg fault". 


\end{document}
