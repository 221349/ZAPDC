\documentclass{article}
\usepackage{a4wide}


\usepackage{polski}
\usepackage[utf8x]{inputenc}
\usepackage{graphicx}
\usepackage{float}
\usepackage{hyperref}
\usepackage{listings}
\usepackage{mathtools}
\usepackage{amsmath}
\usepackage{animate}

\usepackage{color} %red, green, blue, yellow, cyan, magenta, black, white
\definecolor{mygreen}{RGB}{28,172,0} % color values Red, Green, Blue
\definecolor{taupe}{rgb}{0.28, 0.24, 0.2}
\definecolor{mylilas}{RGB}{0,110,0}


\author{Lev Sergeyev}
\title{ZAPDC. Ćwiczenie 1. Aliasing 2D}

\date{2019.03.06}
\begin{document}

\maketitle

%\pagebreakmovie15

\section{Przebieg ćwiczenia}


\begin{figure}[h]
\centering
%\includemovie{3cm}{3cm}{A.gif}
\animategraphics{0.2}{A-}{0}{59}
\caption{Schemat symulacji obiektu K(s) w Simulink}
\end{figure}


\par
Dany jest obiekt o transmitancji:
\begin{equation}
K(s)=\frac{1}{s^2+as+b} 
\end{equation}
\par
Dobierając parametry a, b można uzyskać róźne bieguny, które można podzielić na bieguny stabilne,  niestabilne i bieguny na granicy stabilności. Dodatkowo trzy wyżej wymienione rodzaje można podzielić na rzeczywiste i zespolone, które wprowadzają oscylacje do odpowiedzi.
Dobierając parametry a i b otzrymałem 6 róźnych par biegónów: 
\begin{itemize}  
	\item rzeczywiste:
	\begin{itemize}  
		\item wszystkie ujemne (a=3, b=2)
		\item przynajmniej jeden zerowy (a=1, b=0)
		\item przynajmniej jeden dodatni (a=0, b=-4)
	\end{itemize}
	\item zespolone:
	\begin{itemize}  
		\item wszystkie ujemne (a=0.5, b=1.05)
		\item część rzeczywista jest zero (a=0, b=1)
		\item dodatnie (a=-0.5, b=1.05)
	\end{itemize}
\end{itemize}
Korzystając z śródowiska Simulink przeprowadziłem symulację odpowiedzi skokowych używając schematu na Rys. 1



\pagebreak
\section{Kod źródłowy}



%\lstset{basicstyle=\small}


\lstset{language=Matlab,%
    %basicstyle=\color{red},
    breaklines=true,%
    morekeywords={matlab2tikz},
    keywordstyle=\color{blue},%
    morekeywords=[2]{1}, keywordstyle=[2]{\color{black}},
    identifierstyle=\color{black},%
    stringstyle=\color{mylilas},
    commentstyle=\color{mygreen},%
    showstringspaces=false,%without this there will be a symbol in the places where there is a space
    numbers=left,%
    numberstyle={\tiny \color{taupe}},% size of the numbers
    numbersep=9pt, % this defines how far the numbers are from the text
    emph=[1]{for,end,break},emphstyle=[1]\color{red}, %some words to emphasise
    %emph=[2]{word1,word2}, emphstyle=[2]{style},    
}

\UseRawInputEncoding
\lstinputlisting{../lab1.m}

\end{document}
